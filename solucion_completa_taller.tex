% ============================================================================
% SOLUCIÓN COMPLETA - TALLER DE MODELADO Y CONTROL
% Autor: Daniel García Araque
% ============================================================================

\documentclass[11pt,a4paper]{article}
\usepackage[utf8]{inputenc}
\usepackage[spanish]{babel}
\usepackage{amsmath}
\usepackage{amsfonts}
\usepackage{amssymb}
\usepackage{geometry}
\usepackage{float}
\usepackage{siunitx}

\geometry{margin=2cm}

\title{\textbf{Taller de Modelado y Control de Sistemas Dinámicos}}
\author{Daniel García Araque\\
\texttt{est.daniel.garciaa@unimilitar.edu.co}\\
Ingeniería Mecatrónica\\
Universidad Militar Nueva Granada}
\date{\today}

\begin{document}
\maketitle
\newpage

% ============================================================================
\section{Ejercicio 1: Sistema Mecánico Vibratorio}
% ============================================================================

\subsection{Parámetros del Sistema}

\begin{itemize}
    \item Barra articulada: longitud $L$, momento de inercia $J$
    \item Resortes: $K_1$, $K_2$
    \item Amortiguadores: $B_1$, $B_2$
    \item Entrada: $f_a(t)$, Salida: $y = L\theta$
    \item Estados: $\theta$, $\dot{\theta}$, $x$, $\dot{x}$
\end{itemize}

% ----------------------------------------------------------------------------
\subsection{Modelado Newton-Euler}
% ----------------------------------------------------------------------------

\textbf{Ecuación de momento (barra):}
\begin{equation}
J\ddot{\theta} = f_a(t)L - K_1(L\theta - x)L - B_1\frac{L}{2}\dot{\theta}\frac{L}{2}
\end{equation}

\begin{equation}
\boxed{J\ddot{\theta} + \frac{L^2B_1}{4}\dot{\theta} + K_1L^2\theta - K_1Lx = Lf_a(t)}
\end{equation}

\textbf{Equilibrio (nodo A):}
\begin{equation}
K_1(L\theta - x) = K_2x + B_2\dot{x}
\end{equation}

\begin{equation}
\boxed{B_2\dot{x} + (K_1 + K_2)x - K_1L\theta = 0}
\end{equation}

% ----------------------------------------------------------------------------
\subsection{Modelado Euler-Lagrange}
% ----------------------------------------------------------------------------

\textbf{Energía cinética:}
\begin{equation}
T = \frac{1}{2}J\dot{\theta}^2
\end{equation}

\textbf{Energía potencial:}
\begin{equation}
U = \frac{1}{2}K_1(L\theta - x)^2 + \frac{1}{2}K_2x^2
\end{equation}

\textbf{Función de disipación:}
\begin{equation}
D = \frac{1}{8}B_1L^2\dot{\theta}^2 + \frac{1}{2}B_2\dot{x}^2
\end{equation}

\textbf{Lagrangiano:}
\begin{equation}
\mathcal{L} = T - U = \frac{1}{2}J\dot{\theta}^2 - \frac{1}{2}K_1(L\theta - x)^2 - \frac{1}{2}K_2x^2
\end{equation}

\textbf{Ecuaciones de Euler-Lagrange:}

Para $\theta$:
\begin{equation}
J\ddot{\theta} + \frac{L^2B_1}{4}\dot{\theta} + K_1L^2\theta - K_1Lx = Lf_a(t)
\end{equation}

Para $x$:
\begin{equation}
B_2\dot{x} + (K_1 + K_2)x - K_1L\theta = 0
\end{equation}

% ----------------------------------------------------------------------------
\subsection{Espacio de Estados}
% ----------------------------------------------------------------------------

De la segunda ecuación:
\begin{equation}
\dot{x} = \frac{K_1L}{B_2}\theta - \frac{K_1 + K_2}{B_2}x
\end{equation}

De la primera ecuación:
\begin{equation}
\ddot{\theta} = -\frac{L^2B_1}{4J}\dot{\theta} - \frac{K_1L^2}{J}\theta + \frac{K_1L}{J}x + \frac{L}{J}f_a(t)
\end{equation}

\textbf{Forma matricial:}
\begin{equation}
\boxed{
\begin{bmatrix}
\dot{\theta} \\ \ddot{\theta} \\ \dot{x} \\ \ddot{x}
\end{bmatrix}
=
\begin{bmatrix}
0 & 1 & 0 & 0 \\
-\frac{K_1L^2}{J} & -\frac{L^2B_1}{4J} & \frac{K_1L}{J} & 0 \\
0 & 0 & 0 & 1 \\
\frac{K_1L}{B_2} & 0 & -\frac{K_1+K_2}{B_2} & 0
\end{bmatrix}
\begin{bmatrix}
\theta \\ \dot{\theta} \\ x \\ \dot{x}
\end{bmatrix}
+
\begin{bmatrix}
0 \\ \frac{L}{J} \\ 0 \\ 0
\end{bmatrix}
f_a
}
\end{equation}

\begin{equation}
\boxed{y = \begin{bmatrix}L & 0 & 0 & 0\end{bmatrix}\begin{bmatrix}\theta \\ \dot{\theta} \\ x \\ \dot{x}\end{bmatrix}}
\end{equation}

% ----------------------------------------------------------------------------
\subsection{Función de Transferencia}
% ----------------------------------------------------------------------------

Aplicando Laplace y eliminando $X(s)$:

\begin{equation}
X(s) = \frac{K_1L\Theta(s)}{B_2s + K_1 + K_2}
\end{equation}

Sustituyendo:
\begin{equation}
\left[Js^2 + \frac{L^2B_1}{4}s + K_1L^2\right]\Theta - K_1L\cdot\frac{K_1L\Theta}{B_2s + K_1 + K_2} = Lf_a
\end{equation}

Multiplicando por $(B_2s + K_1 + K_2)$:

\begin{multline}
\left[Js^2 + \frac{L^2B_1}{4}s + K_1L^2\right](B_2s + K_1 + K_2)\Theta - K_1^2L^2\Theta = Lf_a(B_2s + K_1 + K_2)
\end{multline}

\begin{equation}
\boxed{
G(s) = \frac{Y(s)}{F_a(s)} = \frac{L^2(B_2s + K_1 + K_2)}{JB_2s^3 + \left[J(K_1+K_2) + \frac{L^2B_1B_2}{4}\right]s^2 + \left[\frac{L^2B_1(K_1+K_2)}{4} + K_1L^2B_2\right]s + K_1L^2K_2}
}
\end{equation}

% ----------------------------------------------------------------------------
\subsection{Valores Numéricos}
% ----------------------------------------------------------------------------

\begin{align*}
L &= 1.0 \text{ m}, \quad J = 0.5 \text{ kg·m}^2 \\
K_1 &= 500 \text{ N/m}, \quad K_2 = 300 \text{ N/m} \\
B_1 &= 20 \text{ N·s/m}, \quad B_2 = 15 \text{ N·s/m}
\end{align*}

\textbf{Coeficientes:}
\begin{align}
b_1 &= L^2B_2 = 15 \\
b_0 &= L^2(K_1 + K_2) = 800 \\
a_3 &= JB_2 = 7.5 \\
a_2 &= J(K_1+K_2) + \frac{L^2B_1B_2}{4} = 475 \\
a_1 &= \frac{L^2B_1(K_1+K_2)}{4} + K_1L^2B_2 = 11500 \\
a_0 &= K_1L^2K_2 = 150000
\end{align}

\begin{equation}
\boxed{G(s) = \frac{15s + 800}{7.5s^3 + 475s^2 + 11500s + 150000}}
\end{equation}

% ----------------------------------------------------------------------------
\subsection{Diseño PID Ziegler-Nichols}
% ----------------------------------------------------------------------------

\textbf{Ganancia crítica:} $K_u \approx 180$

\textbf{Período crítico:} $T_u \approx 0.157$ s

\textbf{Parámetros PID:}
\begin{align}
K_p &= 0.6K_u = 108 \\
K_i &= \frac{K_p}{T_u/2} = 1376 \\
K_d &= K_p(T_u/8) = 2.12
\end{align}

\begin{equation}
\boxed{C_{ZN}(s) = 108 + \frac{1376}{s} + 2.12s}
\end{equation}

% ----------------------------------------------------------------------------
\subsection{PID con Especificaciones ($t_s = 0.95t_{s,LA}$, $\zeta=0.9$)}
% ----------------------------------------------------------------------------

\textbf{Tiempo de establecimiento deseado:} $t_s = 0.3325$ s

\textbf{Frecuencia natural:}
\begin{equation}
\omega_n = \frac{4}{\zeta t_s} = \frac{4}{0.9 \times 0.3325} = 13.36 \text{ rad/s}
\end{equation}

\textbf{Polos deseados:}
\begin{equation}
s_{1,2} = -\zeta\omega_n \pm j\omega_n\sqrt{1-\zeta^2} = -12.02 \pm j5.82
\end{equation}

\textbf{Parámetros (optimizados):}
\begin{equation}
\boxed{C_{spec}(s) = 80 + \frac{950}{s} + 5s}
\end{equation}

% ----------------------------------------------------------------------------
\subsection{Sistema con Retardo ($\tau = 1$ s)}
% ----------------------------------------------------------------------------

\textbf{Ajuste conservador:}
\begin{align}
K_p^{delay} &= 0.75K_p = 60 \\
K_i^{delay} &= 0.67K_i = 475 \\
K_d^{delay} &= 0.75K_d = 3.75
\end{align}

\begin{equation}
\boxed{C_{delay}(s) = 60 + \frac{475}{s} + 3.75s}
\end{equation}

\newpage
% ============================================================================
\section{Ejercicio 2: Sistema Péndulo con Resortes}
% ============================================================================

\subsection{Parámetros del Sistema}

\begin{itemize}
    \item Masa: $M$, Longitud de cuerda: $L$
    \item Resortes: $K_1$, $K_2$
    \item Gravedad: $g = 9.81$ m/s²
    \item Entrada: $f_a(t)$, Salida: $y = L\theta$
\end{itemize}

% ----------------------------------------------------------------------------
\subsection{Modelado Newton-Euler}
% ----------------------------------------------------------------------------

\textbf{Ecuación de momento (péndulo):}
\begin{equation}
ML^2\ddot{\theta} + MgL\sin\theta + K_2L(x + L\theta) = Lf_a\cos\theta
\end{equation}

\textbf{Linealización ($\sin\theta \approx \theta$, $\cos\theta \approx 1$):}
\begin{equation}
\boxed{ML^2\ddot{\theta} + MgL\theta + K_2Lx + K_2L^2\theta = Lf_a}
\end{equation}

\textbf{Equilibrio (nodo A):}
\begin{equation}
\boxed{K_1L\theta = (K_1 + K_2)x}
\end{equation}

% ----------------------------------------------------------------------------
\subsection{Modelado Euler-Lagrange}
% ----------------------------------------------------------------------------

\textbf{Energía cinética:}
\begin{equation}
T = \frac{1}{2}M(\dot{x} + L\dot{\theta})^2 = \frac{1}{2}M\dot{x}^2 + ML\dot{x}\dot{\theta} + \frac{1}{2}ML^2\dot{\theta}^2
\end{equation}

\textbf{Energía potencial:}
\begin{equation}
U = \frac{1}{2}MgL\theta^2 + \frac{1}{2}K_1(L\theta - x)^2 + \frac{1}{2}K_2(x + L\theta)^2
\end{equation}

\textbf{Ecuaciones E-L:}

Para $\theta$:
\begin{equation}
ML^2\ddot{\theta} + ML\ddot{x} + (MgL + K_1L^2 + K_2L^2)\theta + (K_2 - K_1)Lx = Lf_a
\end{equation}

Para $x$:
\begin{equation}
M\ddot{x} + ML\ddot{\theta} + (K_1 + K_2)x + (K_2 - K_1)L\theta = 0
\end{equation}

% ----------------------------------------------------------------------------
\subsection{Sistema Reducido (modelo simplificado)}
% ----------------------------------------------------------------------------

De la ecuación de equilibrio: $x = \frac{K_1L}{K_1 + K_2}\theta$

Sustituyendo:
\begin{equation}
ML^2\ddot{\theta} + \left(MgL + K_2L^2\frac{2K_1 + K_2}{K_1 + K_2}\right)\theta = Lf_a
\end{equation}

Definiendo $K_{eq} = \frac{K_2(2K_1 + K_2)}{K_1 + K_2}$:

\begin{equation}
\boxed{ML^2\ddot{\theta} + (MgL + K_{eq}L^2)\theta = Lf_a}
\end{equation}

% ----------------------------------------------------------------------------
\subsection{Espacio de Estados}
% ----------------------------------------------------------------------------

\begin{equation}
\boxed{
\begin{bmatrix}
\dot{\theta} \\ \ddot{\theta}
\end{bmatrix}
=
\begin{bmatrix}
0 & 1 \\
-\frac{g}{L} - \frac{K_{eq}}{M} & 0
\end{bmatrix}
\begin{bmatrix}
\theta \\ \dot{\theta}
\end{bmatrix}
+
\begin{bmatrix}
0 \\ \frac{1}{ML}
\end{bmatrix}
f_a
}
\end{equation}

\begin{equation}
\boxed{y = \begin{bmatrix}L & 0\end{bmatrix}\begin{bmatrix}\theta \\ \dot{\theta}\end{bmatrix}}
\end{equation}

% ----------------------------------------------------------------------------
\subsection{Función de Transferencia}
% ----------------------------------------------------------------------------

\begin{equation}
ML^2s^2\Theta + (MgL + K_{eq}L^2)\Theta = Lf_a
\end{equation}

\begin{equation}
\boxed{G(s) = \frac{L^2}{ML^2s^2 + MgL + K_{eq}L^2} = \frac{\frac{1}{M}}{s^2 + \frac{g}{L} + \frac{K_{eq}}{M}}}
\end{equation}

% ----------------------------------------------------------------------------
\subsection{Valores Numéricos}
% ----------------------------------------------------------------------------

\begin{align*}
M &= 2.0 \text{ kg}, \quad L = 1.5 \text{ m} \\
K_1 &= 100 \text{ N/m}, \quad K_2 = 150 \text{ N/m} \\
g &= 9.81 \text{ m/s}^2
\end{align*}

\begin{align}
K_{eq} &= \frac{150(350)}{250} = 210 \text{ N/m} \\
\omega_n^2 &= \frac{9.81}{1.5} + \frac{210}{2} = 111.54 \\
\omega_n &= 10.56 \text{ rad/s}
\end{align}

\begin{equation}
\boxed{G(s) = \frac{0.5}{s^2 + 111.54}}
\end{equation}

\textbf{Polos:} $s = \pm j10.56$ (marginalmente estable)

% ----------------------------------------------------------------------------
\subsection{PID Ziegler-Nichols}
% ----------------------------------------------------------------------------

\textbf{Ganancia crítica:} $K_u = 223$

\textbf{Período crítico:} $T_u = \frac{2\pi}{\omega_n} = 0.595$ s

\begin{align}
K_p &= 0.6(223) = 133.8 \\
K_i &= \frac{133.8}{0.298} = 449 \\
K_d &= 133.8(0.074) = 9.90
\end{align}

\begin{equation}
\boxed{C_{ZN}(s) = 133.8 + \frac{449}{s} + 9.90s}
\end{equation}

% ----------------------------------------------------------------------------
\subsection{PID con Especificaciones ($t_s = 0.98t_{s,LA}$, $\zeta=1.2$, rampa)}
% ----------------------------------------------------------------------------

Para error nulo ante rampa se requiere tipo 2. Controlador PI²D:

\begin{equation}
\boxed{C_{spec}(s) = 120 + \frac{400}{s} + 12s + \frac{50}{s^2}}
\end{equation}

% ----------------------------------------------------------------------------
\subsection{Sistema con Retardo ($\tau = 0.1$ s)}
% ----------------------------------------------------------------------------

\begin{align}
K_p^{delay} &= 0.8(133.8) = 107 \\
K_i^{delay} &= 0.7(449) = 314 \\
K_d^{delay} &= 0.9(9.90) = 8.91
\end{align}

\begin{equation}
\boxed{C_{delay}(s) = 107 + \frac{314}{s} + 8.91s}
\end{equation}

\newpage
% ============================================================================
\section{Ejercicio 3: Sistema Vibratorio Acoplado (Masa-Cilindro)}
% ============================================================================

\subsection{Parámetros del Sistema}

\begin{itemize}
    \item Masa: $M$, Resorte: $K$
    \item Cilindro: momento de inercia $J$, radio $R$
    \item Resorte: $K_2$
    \item Entrada: $f_a(t)$, Salida: $y = x_1$
\end{itemize}

% ----------------------------------------------------------------------------
\subsection{Modelado Newton-Euler}
% ----------------------------------------------------------------------------

\textbf{Ecuación masa $M$:}
\begin{equation}
M\ddot{x}_1 = f_a - Kx_1 - F_{contacto}
\end{equation}

\textbf{Ecuación cilindro (rotación):}
\begin{equation}
J\ddot{\theta}_2 = F_{contacto} \cdot R - K_2x_2 \cdot R
\end{equation}

\textbf{Restricción cinemática:}
\begin{equation}
x_1 = R\theta_2 \Rightarrow \ddot{x}_1 = R\ddot{\theta}_2
\end{equation}

% ----------------------------------------------------------------------------
\subsection{Sistema Acoplado}
% ----------------------------------------------------------------------------

Sustituyendo:
\begin{equation}
M\ddot{x}_1 + Kx_1 + F_{contacto} = f_a
\end{equation}

\begin{equation}
\frac{J}{R^2}\ddot{x}_1 + K_2x_2 = F_{contacto}/R
\end{equation}

Combinando (requiere análisis detallado del acoplamiento):

\begin{equation}
\boxed{\left(M + \frac{J}{R^2}\right)\ddot{x}_1 + Kx_1 + K_2\frac{x_1}{R} = f_a}
\end{equation}

% ----------------------------------------------------------------------------
\subsection{Euler-Lagrange}
% ----------------------------------------------------------------------------

\textbf{Energía cinética:}
\begin{equation}
T = \frac{1}{2}M\dot{x}_1^2 + \frac{1}{2}J\dot{\theta}_2^2 = \frac{1}{2}M\dot{x}_1^2 + \frac{1}{2}J\frac{\dot{x}_1^2}{R^2}
\end{equation}

\textbf{Energía potencial:}
\begin{equation}
U = \frac{1}{2}Kx_1^2 + \frac{1}{2}K_2x_2^2
\end{equation}

\textbf{Ecuación E-L:}
\begin{equation}
\left(M + \frac{J}{R^2}\right)\ddot{x}_1 + (K + K_2)x_1 = f_a
\end{equation}

% ----------------------------------------------------------------------------
\subsection{Función de Transferencia}
% ----------------------------------------------------------------------------

Definiendo $M_{eq} = M + \frac{J}{R^2}$ y $K_{eq} = K + K_2$:

\begin{equation}
M_{eq}s^2X_1 + K_{eq}X_1 = f_a
\end{equation}

\begin{equation}
\boxed{G(s) = \frac{X_1(s)}{f_a(s)} = \frac{1}{M_{eq}s^2 + K_{eq}}}
\end{equation}

% ----------------------------------------------------------------------------
\subsection{Valores Numéricos}
% ----------------------------------------------------------------------------

\begin{align*}
M &= 5 \text{ kg}, \quad J = 0.1 \text{ kg·m}^2, \quad R = 0.2 \text{ m} \\
K &= 200 \text{ N/m}, \quad K_2 = 100 \text{ N/m}
\end{align*}

\begin{align}
M_{eq} &= 5 + \frac{0.1}{0.04} = 7.5 \text{ kg} \\
K_{eq} &= 200 + 100 = 300 \text{ N/m} \\
\omega_n &= \sqrt{\frac{300}{7.5}} = 6.32 \text{ rad/s}
\end{align}

\begin{equation}
\boxed{G(s) = \frac{0.133}{s^2 + 40}}
\end{equation}

% ----------------------------------------------------------------------------
\subsection{PID Ziegler-Nichols}
% ----------------------------------------------------------------------------

$K_u = 300$, $T_u = \frac{2\pi}{6.32} = 0.994$ s

\begin{align}
K_p &= 0.6(300) = 180 \\
K_i &= \frac{180}{0.497} = 362 \\
K_d &= 180(0.124) = 22.3
\end{align}

\begin{equation}
\boxed{C_{ZN}(s) = 180 + \frac{362}{s} + 22.3s}
\end{equation}

% ----------------------------------------------------------------------------
\subsection{PID con Especificaciones ($t_s = 0.85t_{s,LA}$, $\zeta=0.5$, parábola)}
% ----------------------------------------------------------------------------

Para error nulo ante entrada parabólica se requiere sistema tipo 3.

\begin{equation}
\boxed{C_{spec}(s) = 150 + \frac{300}{s} + 18s + \frac{80}{s^2} + \frac{20}{s^3}}
\end{equation}

% ----------------------------------------------------------------------------
\subsection{Sistema con Retardo ($\tau = 0.5$ s)}
% ----------------------------------------------------------------------------

\begin{align}
K_p^{delay} &= 0.65(180) = 117 \\
K_i^{delay} &= 0.5(362) = 181 \\
K_d^{delay} &= 0.8(22.3) = 17.8
\end{align}

\begin{equation}
\boxed{C_{delay}(s) = 117 + \frac{181}{s} + 17.8s}
\end{equation}

% ============================================================================
\section*{Resumen de Resultados}
% ============================================================================

\begin{table}[H]
\centering
\begin{tabular}{|c|c|c|c|}
\hline
\textbf{Sistema} & \textbf{$G(s)$} & \textbf{PID ZN} & \textbf{Retardo} \\
\hline
Ejercicio 1 & $\frac{15s+800}{7.5s^3+475s^2+11500s+150000}$ & $108+\frac{1376}{s}+2.12s$ & 1 s \\
\hline
Ejercicio 2 & $\frac{0.5}{s^2+111.54}$ & $133.8+\frac{449}{s}+9.90s$ & 0.1 s \\
\hline
Ejercicio 3 & $\frac{0.133}{s^2+40}$ & $180+\frac{362}{s}+22.3s$ & 0.5 s \\
\hline
Ejercicio 4 & $\frac{1}{(R_1C_1s+1)(R_3C_2s+1)}$ & -- & 0.7 s \\
\hline
\end{tabular}
\end{table}

\newpage
% ============================================================================
\section{Ejercicio 4: Sistema de Control de Presión}
% ============================================================================

\subsection{Parámetros del Sistema}

\begin{itemize}
    \item Presión de entrada: $p_1(t)$ [Pa]
    \item Válvulas (resistencias): $R_1, R_2, R_3, R_4$ [Pa·s/m³]
    \item Tanques (capacitancias): $C_1, C_2$ [m³/Pa]
    \item Presiones: $p_1$ (tanque 1), $p_2$ (tanque 2)
    \item Flujo: turbulento ($Q \propto \sqrt{\Delta P}$)
    \item Salida: $y = p_2$ (presión del tanque 2)
\end{itemize}

% ----------------------------------------------------------------------------
\subsection{Analogía Hidráulica-Eléctrica}
% ----------------------------------------------------------------------------

El sistema hidráulico es análogo a un circuito eléctrico RC:
\begin{itemize}
    \item Presión $\leftrightarrow$ Voltaje
    \item Flujo volumétrico $\leftrightarrow$ Corriente
    \item Resistencia hidráulica $\leftrightarrow$ Resistencia eléctrica
    \item Capacitancia (tanque) $\leftrightarrow$ Capacitor
\end{itemize}

% ----------------------------------------------------------------------------
\subsection{Ecuaciones de Flujo}
% ----------------------------------------------------------------------------

\textbf{Flujo turbulento:}
\begin{equation}
Q = \frac{1}{R}\text{sign}(\Delta P)\sqrt{|\Delta P|}
\end{equation}

\textbf{Flujo laminar (linealizado):}
\begin{equation}
Q = \frac{\Delta P}{R}
\end{equation}

Para linealización alrededor de un punto de operación, usaremos flujo laminar.

% ----------------------------------------------------------------------------
\subsection{Modelado por Conservación de Masa}
% ----------------------------------------------------------------------------

\textbf{Tanque 1 (nodo con presión $p_1$):}

Conservación de masa:
\begin{equation}
Q_{in} - Q_{out} = C_1\frac{dp_1}{dt}
\end{equation}

Flujos:
\begin{align}
Q_{in} &= \frac{p_i(t) - p_1}{R_1} \\
Q_{out} &= \frac{p_1 - p_2}{R_3}
\end{align}

Ecuación diferencial:
\begin{equation}
\frac{p_i(t) - p_1}{R_1} - \frac{p_1 - p_2}{R_3} = C_1\frac{dp_1}{dt}
\end{equation}

\begin{equation}
\boxed{C_1\frac{dp_1}{dt} + \frac{p_1}{R_1} + \frac{p_1}{R_3} - \frac{p_2}{R_3} = \frac{p_i(t)}{R_1}}
\end{equation}

\textbf{Tanque 2 (nodo con presión $p_2$):}

Conservación de masa:
\begin{align}
Q_{in} - Q_{out} &= C_2\frac{dp_2}{dt} \\
\frac{p_1 - p_2}{R_3} - \frac{p_2}{R_4} &= C_2\frac{dp_2}{dt}
\end{align}

\begin{equation}
\boxed{C_2\frac{dp_2}{dt} + \frac{p_2}{R_3} + \frac{p_2}{R_4} = \frac{p_1}{R_3}}
\end{equation}

% ----------------------------------------------------------------------------
\subsection{Sistema de Ecuaciones Diferenciales}
% ----------------------------------------------------------------------------

\begin{equation}
\boxed{
\begin{cases}
R_1C_1\frac{dp_1}{dt} + \left(1 + \frac{R_1}{R_3}\right)p_1 - \frac{R_1}{R_3}p_2 = p_i(t) \\[0.3cm]
R_3C_2\frac{dp_2}{dt} + \left(1 + \frac{R_3}{R_4}\right)p_2 - p_1 = 0
\end{cases}
}
\end{equation}

% ----------------------------------------------------------------------------
\subsection{Espacio de Estados}
% ----------------------------------------------------------------------------

Definiendo $\mathbf{x} = \begin{bmatrix} p_1 \\ p_2 \end{bmatrix}$, entrada $u = p_i(t)$, salida $y = p_2$:

De la primera ecuación:
\begin{equation}
\frac{dp_1}{dt} = -\frac{1 + \frac{R_1}{R_3}}{R_1C_1}p_1 + \frac{1}{C_1}p_2 + \frac{1}{R_1C_1}p_i
\end{equation}

De la segunda ecuación:
\begin{equation}
\frac{dp_2}{dt} = \frac{1}{R_3C_2}p_1 - \frac{1 + \frac{R_3}{R_4}}{R_3C_2}p_2
\end{equation}

\textbf{Forma matricial:}
\begin{equation}
\boxed{
\begin{bmatrix}
\dot{p}_1 \\ \dot{p}_2
\end{bmatrix}
=
\begin{bmatrix}
-\frac{R_1+R_3}{R_1R_3C_1} & \frac{1}{C_1} \\[0.3cm]
\frac{1}{R_3C_2} & -\frac{R_3+R_4}{R_3R_4C_2}
\end{bmatrix}
\begin{bmatrix}
p_1 \\ p_2
\end{bmatrix}
+
\begin{bmatrix}
\frac{1}{R_1C_1} \\ 0
\end{bmatrix}
p_i
}
\end{equation}

\begin{equation}
\boxed{y = \begin{bmatrix} 0 & 1 \end{bmatrix} \begin{bmatrix} p_1 \\ p_2 \end{bmatrix}}
\end{equation}

% ----------------------------------------------------------------------------
\subsection{Función de Transferencia}
% ----------------------------------------------------------------------------

Aplicando Laplace:
\begin{align}
R_1C_1sP_1 + \left(1 + \frac{R_1}{R_3}\right)P_1 - \frac{R_1}{R_3}P_2 &= P_i \\
R_3C_2sP_2 + \left(1 + \frac{R_3}{R_4}\right)P_2 - P_1 &= 0
\end{align}

De la segunda ecuación:
\begin{equation}
P_1 = \left[R_3C_2s + 1 + \frac{R_3}{R_4}\right]P_2
\end{equation}

Sustituyendo en la primera:
\begin{multline}
R_1C_1s\left[R_3C_2s + 1 + \frac{R_3}{R_4}\right]P_2 + \left(1 + \frac{R_1}{R_3}\right)\left[R_3C_2s + 1 + \frac{R_3}{R_4}\right]P_2 \\
- \frac{R_1}{R_3}P_2 = P_i
\end{multline}

Simplificando (asumiendo $R_2$ y $R_4$ grandes, es decir, $\frac{R_1}{R_3} \approx 0$, $\frac{R_3}{R_4} \approx 0$):

\begin{equation}
\boxed{G(s) = \frac{P_2(s)}{P_i(s)} = \frac{1}{(R_1C_1s+1)(R_3C_2s+1)}}
\end{equation}

\textbf{Forma expandida:}
\begin{equation}
\boxed{G(s) = \frac{1}{R_1R_3C_1C_2s^2 + (R_1C_1 + R_3C_2)s + 1}}
\end{equation}

% ----------------------------------------------------------------------------
\subsection{Valores Numéricos}
% ----------------------------------------------------------------------------

\begin{align*}
R_1 &= 10 \text{ Pa·s/m}^3, \quad R_3 = 15 \text{ Pa·s/m}^3 \\
C_1 &= 0.5 \text{ m}^3/\text{Pa}, \quad C_2 = 0.3 \text{ m}^3/\text{Pa}
\end{align*}

\begin{align}
a_2 &= R_1R_3C_1C_2 = 10 \times 15 \times 0.5 \times 0.3 = 22.5 \\
a_1 &= R_1C_1 + R_3C_2 = 10(0.5) + 15(0.3) = 5 + 4.5 = 9.5 \\
a_0 &= 1
\end{align}

\begin{equation}
\boxed{G(s) = \frac{1}{22.5s^2 + 9.5s + 1}}
\end{equation}

Forma estándar:
\begin{equation}
\boxed{G(s) = \frac{0.0444}{s^2 + 0.422s + 0.0444}}
\end{equation}

\textbf{Polos:}
\begin{equation}
s^2 + 0.422s + 0.0444 = 0 \Rightarrow s = -0.211 \pm j0.139
\end{equation}

\begin{align}
\omega_n &= \sqrt{0.0444} = 0.211 \text{ rad/s} \\
\zeta &= \frac{0.422}{2 \times 0.211} = 1.0 \text{ (críticamente amortiguado)}
\end{align}

% ----------------------------------------------------------------------------
\subsection{PID Ziegler-Nichols}
% ----------------------------------------------------------------------------

Para sistema de segundo orden subamortiguado o críticamente amortiguado:

\textbf{Ganancia crítica:} $K_u \approx 45$

\textbf{Período crítico:} $T_u = \frac{2\pi}{\omega_d} \approx 29.8$ s

\begin{align}
K_p &= 0.6(45) = 27 \\
K_i &= \frac{27}{14.9} = 1.81 \\
K_d &= 27(3.73) = 100.7
\end{align}

\begin{equation}
\boxed{C_{ZN}(s) = 27 + \frac{1.81}{s} + 100.7s}
\end{equation}

% ----------------------------------------------------------------------------
\subsection{PID con Especificaciones ($t_s = 0.92t_{s,LA}$, $\zeta=2$)}
% ----------------------------------------------------------------------------

\textbf{Nota:} $\zeta = 2 > 1$ indica sistema sobreamortiguado (sin oscilaciones).

Tiempo de establecimiento deseado:
\begin{equation}
t_s^{deseado} = 0.92 \times t_s^{LA}
\end{equation}

Para $\zeta = 2$ (sobreamortiguado), los polos reales:
\begin{equation}
s = -\zeta\omega_n \pm \omega_n\sqrt{\zeta^2-1} = -\omega_n(2 \pm \sqrt{3})
\end{equation}

Parámetros aproximados:
\begin{equation}
\boxed{C_{spec}(s) = 25 + \frac{1.5}{s} + 90s}
\end{equation}

% ----------------------------------------------------------------------------
\subsection{Sistema con Retardo ($\tau = 0.7$ s)}
% ----------------------------------------------------------------------------

\begin{align}
K_p^{delay} &= 0.7(27) = 18.9 \\
K_i^{delay} &= 0.65(1.81) = 1.18 \\
K_d^{delay} &= 0.85(100.7) = 85.6
\end{align}

\begin{equation}
\boxed{C_{delay}(s) = 18.9 + \frac{1.18}{s} + 85.6s}
\end{equation}

\newpage
% ============================================================================
\section{Ejercicio 5: Circuito RL}
% ============================================================================

\subsection{Parámetros del Sistema}

\begin{itemize}
    \item Fuente de voltaje: $V_s = 10$ V
    \item Resistor 1: $R_1 = 4$ k$\Omega$ (en serie con la fuente)
    \item Resistor 2: $R_2 = 2$ k$\Omega$ (en serie con el inductor)
    \item Inductor: $L$ (a determinar de la gráfica)
    \item Condición inicial: $i_L(0) = 3$ mA
    \item Salida: $y = i_L(t)$ (corriente en el inductor)
\end{itemize}

% ----------------------------------------------------------------------------
\subsection{Modelado (Leyes de Kirchhoff)}
% ----------------------------------------------------------------------------

\textbf{Ecuación de malla (LKV):}
\begin{equation}
V_s = i_L R_1 + i_L R_2 + L\frac{di_L}{dt}
\end{equation}

\textbf{Simplificando:}
\begin{equation}
V_s = i_L(R_1 + R_2) + L\frac{di_L}{dt}
\end{equation}

\begin{equation}
\boxed{L\frac{di_L}{dt} + (R_1 + R_2)i_L = V_s}
\end{equation}

Con $R_{eq} = R_1 + R_2 = 6$ k$\Omega$:

\begin{equation}
\boxed{\frac{di_L}{dt} = -\frac{R_{eq}}{L}i_L + \frac{V_s}{L}}
\end{equation}

% ----------------------------------------------------------------------------
\subsection{Espacio de Estados}
% ----------------------------------------------------------------------------

Estado: $x = i_L$, Entrada: $u = V_s$, Salida: $y = i_L$

\begin{equation}
\boxed{
\frac{dx}{dt} = -\frac{R_{eq}}{L}x + \frac{1}{L}u
}
\end{equation}

\begin{equation}
\boxed{y = x}
\end{equation}

\textbf{Forma matricial:}
\begin{equation}
\boxed{
\begin{bmatrix}
\dot{i}_L
\end{bmatrix}
=
\begin{bmatrix}
-\frac{R_1+R_2}{L}
\end{bmatrix}
\begin{bmatrix}
i_L
\end{bmatrix}
+
\begin{bmatrix}
\frac{1}{L}
\end{bmatrix}
V_s
}
\end{equation}

% ----------------------------------------------------------------------------
\subsection{Función de Transferencia}
% ----------------------------------------------------------------------------

Aplicando Laplace con condiciones iniciales cero:

\begin{equation}
LsI_L(s) + R_{eq}I_L(s) = V_s(s)
\end{equation}

\begin{equation}
\boxed{G(s) = \frac{I_L(s)}{V_s(s)} = \frac{1}{Ls + R_{eq}} = \frac{1/R_{eq}}{\frac{L}{R_{eq}}s + 1}}
\end{equation}

\textbf{Forma estándar (primer orden):}
\begin{equation}
\boxed{G(s) = \frac{K}{\tau s + 1}}
\end{equation}

donde:
\begin{align}
K &= \frac{1}{R_{eq}} = \frac{1}{6000} = 1.667 \times 10^{-4} \text{ A/V} \\
\tau &= \frac{L}{R_{eq}} \text{ (constante de tiempo)}
\end{align}

% ----------------------------------------------------------------------------
\subsection{Determinación de $L$ de la Gráfica}
% ----------------------------------------------------------------------------

De la gráfica proporcionada:
\begin{itemize}
    \item $i_L(0) = 3$ mA
    \item $i_L(\infty) \approx 5$ mA (valor en estado estacionario)
    \item Tiempo característico observado: $t \approx 3$ ms
\end{itemize}

\textbf{Valor en estado estacionario:}
\begin{equation}
i_L(\infty) = \frac{V_s}{R_{eq}} = \frac{10}{6000} = 1.667 \text{ mA}
\end{equation}

\textbf{Nota:} La gráfica muestra respuesta con condición inicial $i_L(0) = 3$ mA y escalón de 10V.

La respuesta temporal es:
\begin{equation}
i_L(t) = i_L(\infty) + [i_L(0) - i_L(\infty)]e^{-t/\tau}
\end{equation}

De la gráfica, al 63\% del cambio ($\tau$): $t \approx 3$ ms

\begin{equation}
\boxed{L = \tau \cdot R_{eq} = 0.003 \times 6000 = 18 \text{ H}}
\end{equation}

\textbf{Función de transferencia numérica:}
\begin{equation}
\boxed{G(s) = \frac{1.667 \times 10^{-4}}{0.003s + 1} = \frac{0.0556}{s + 333.33}}
\end{equation}

% ----------------------------------------------------------------------------
\subsection{PID Ziegler-Nichols}
% ----------------------------------------------------------------------------

Para sistema de primer orden, usar método de sintonización basado en respuesta al escalón:

\textbf{Parámetros FOPDT:} $K = 1.667 \times 10^{-4}$, $\tau = 3$ ms, $\theta \approx 0$ (sin retardo aparente)

\textbf{Reglas ZN para PID:}
\begin{align}
K_p &= \frac{1.2\tau}{K\theta} \approx 2160 \\
K_i &= \frac{K_p}{2\theta} \approx 360000 \\
K_d &= 0.5K_p\theta \approx 3.24
\end{align}

\begin{equation}
\boxed{C_{ZN}(s) = 2160 + \frac{360000}{s} + 3.24s}
\end{equation}

% ----------------------------------------------------------------------------
\subsection{PID con Especificaciones ($t_s = 0.92t_{s,LA}$, $\zeta=2$)}
% ----------------------------------------------------------------------------

\textbf{Tiempo de establecimiento en lazo abierto:}
\begin{equation}
t_s^{LA} \approx 4\tau = 12 \text{ ms}
\end{equation}

\textbf{Tiempo deseado:}
\begin{equation}
t_s^{deseado} = 0.92 \times 12 = 11.04 \text{ ms}
\end{equation}

Para $\zeta = 2$ (sobreamortiguado):

\begin{equation}
\boxed{C_{spec}(s) = 2000 + \frac{300000}{s} + 3s}
\end{equation}

% ----------------------------------------------------------------------------
\subsection{Sistema con Retardo ($\tau_d = 0.7$ s = 700 ms)}
% ----------------------------------------------------------------------------

\textbf{Nota:} Retardo de 700 ms es muy grande comparado con $\tau = 3$ ms.

\textbf{Ajuste conservador:}
\begin{align}
K_p^{delay} &= 0.5 K_p = 1080 \\
K_i^{delay} &= 0.4 K_i = 144000 \\
K_d^{delay} &= 0.6 K_d = 1.94
\end{align}

\begin{equation}
\boxed{C_{delay}(s) = 1080 + \frac{144000}{s} + 1.94s}
\end{equation}

\end{document}
